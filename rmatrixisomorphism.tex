% ****** Start of file rmatrixisomorphism.tex ******
%
%   This file is prepared for submission to AIP Publishing Journals
%   using REVTeX 4.1 in compliance with AIP style guidelines.
%
% It also requires running BibTeX. The commands are as follows:
%
%  1)  latex  rmatrixisomorphism
%  2)  bibtex rmatrixisomorphism
%  3)  latex  rmatrixisomorphism
%  4)  latex  rmatrixisomorphism
%

\documentclass[%
 aip,
 amsmath,amssymb,
 reprint,%
]{revtex4-1}

\usepackage{graphicx}% Include figure files
\usepackage{dcolumn}% Align table columns on decimal point
\usepackage{bm}% bold math
\usepackage{mathptmx}% Times Roman font
\usepackage{etoolbox}
\usepackage{bbm}
\usepackage{amsthm}% For theorem environments
\usepackage{mathtools}
\usepackage{mathrsfs}

\usepackage[utf8]{inputenc}
\usepackage[T1]{fontenc}

%% AIP requests corresponding email to be moved after affiliations
\makeatletter
\def\@email#1#2{%
 \endgroup
 \patchcmd{\titleblock@produce}
  {\frontmatter@RRAPformat}
  {\frontmatter@RRAPformat{\produce@RRAP{*#1\href{mailto:#2}{#2}}}\frontmatter@RRAPformat}
  {}{}
}%
\makeatother

% Define theorem environments
\theoremstyle{plain}
\newtheorem{theorem}{Theorem}[section]
\newtheorem{lemma}[theorem]{Lemma}
\newtheorem{proposition}[theorem]{Proposition}
\newtheorem{corollary}[theorem]{Corollary}
\theoremstyle{definition}
\newtheorem{definition}[theorem]{Definition}
\newtheorem{remark}[theorem]{Remark}
\newtheorem{example}{Example}[section]

%%%%%%%%%%%%%%%%%%%%%%%%%%%%
% Notation
%%%%%%%%%%%%%%%%%%%%%%%%%%%%
\newcommand{\R}{\mathbb{R}}
\newcommand{\C}{\mathbb{C}}

\begin{document}

\preprint{AIP/123-QED}

\title{A Yang-Baxter Representation of the $\zeta$ Function}

\author{Drew Remmenga}
\email{drewremmenga@gmail.com}
\affiliation{%
Fort Collins, Colorado
}%

\date{\today}% Current date

\maketitle

\begin{quotation}
We study a formal calculus arising from a regularized Weierstrass product
\[
\star(x)=\prod_{n\in\mathbb{Z}} \bigl(x-(2n-1)\pi i\bigr),
\]
whose zero set coincides with that of $\cosh(x/2)$.  By encoding the derivatives
of $\star$ using complete Bell polynomials \cite{bell1934exponential}, we define formal integral transforms
\[
\sigma(s,n,m)=\int_{0}^{\infty} x^{s}\,\star B_{n}(x)\,\star B_{m}(x)\,dx,\qquad
\tau(s,n,m)=\bigl[ x^{s}\star B_{n}\star B_{m}\bigr]_{0}^{\infty},
\]
and derive a closed system of linear recurrences in $(s,n,m)$ by
integration by parts.  These identities exhibit symmetry, a two-step
quasi-periodicity, and parity constraints.  Interpreting $\sigma$ and $\tau$ as
formal matrix elements, we construct an $R$-matrix of Temperley-Lieb type and
prove that it satisfies the Yang-Baxter equation solely as a consequence of the
recurrence system.  All results are formal and do not rely on analytic
convergence.  The correspondence highlights a purely algebraic link between the
calculus of a classical Weierstrass product and the structure of integrable
vertex models.
\end{quotation}


\section{Introduction}

The classical Weierstrass product for $\cosh(x/2)$ \cite{ahlfors1979complex, elizalde1995ten,voros1987spectral} motivates the formal
infinite product
\begin{equation}\label{eq:star_product_intro}
\star(x)=\prod_{n\in\mathbb{Z}} \bigl(x-(2n-1)\pi i\bigr),
\end{equation}
which we treat throughout as a formal object whose zero set agrees with that of
$\cosh(x/2)$.  Writing $\star(x)=C\cosh(x/2)$ with an unspecified constant $C$,
we encode the derivatives of $\star$ using complete Bell polynomials:
\[
\star B_{n}(x)=\frac{d^{n}}{dx^{n}}\star(x)
=\star(x)\,B_{n}\bigl(g'(x),\dots,g^{(n)}(x)\bigr),\qquad g=\log\star.
\]

Using this structure, we define formal transforms
\[
\sigma(s,n,m)=\int_{0}^{\infty} x^{s}\,\star B_{n}\star B_{m}\,dx,\qquad
\tau(s,n,m)=\bigl[x^{s}\star B_{n}\star B_{m}\bigr]_{0}^{\infty},
\]
and show that they satisfy a closed family of recurrence relations in $s,n,m$.
The boundary term $\tau$ vanishes whenever either index is odd and satisfies
two-step quasi-periodicity in the first index.

The demonstrated isomorphism is between $\zeta$ and the base case $\sigma(s,0,0)$.

These properties are reminiscent of the functional identities satisfied by
Boltzmann weights in integrable vertex models.  We make this analogy precise by
constructing a formal $R$-matrix from the transforms and proving that it
satisfies the Yang-Baxter equation.

The results are formal: we do not assume convergence of the integrals defining
$\sigma$ or the existence of the limits defining $\tau$.  Instead, $\sigma$ and
$\tau$ are universal symbols constrained only by the derivative identity
$\frac{d}{dx}(\star B_{n})=\star B_{n+1}$ and the parity structure of $\star$.
This formal viewpoint isolates the algebraic features underlying the
recurrences and reveals a connection to Temperley-Lieb $R$-matrices.


\section{The Formal Weierstrass Product and Its Derivatives}

\subsection{Definition and normalization}

\begin{definition}
Define the truncated product
\[
\star_{N}(x)=\prod_{n=-N}^{N}\bigl(x-(2n-1)\pi i\bigr).
\]
A \emph{regularized Weierstrass product} $\star(x)$ is any formal object
satisfying
\[
\star(x)=C\cosh(x/2)
\]
for a nonzero constant $C$, and whose zero set is the set of odd integer
multiples of $\pi i$.
\end{definition}

Only algebraic properties of $\cosh(x/2)$ will be used; the value of $C$ plays
no role in the recurrence relations.


\subsection{Logarithmic derivatives}

Write
\[
g(x)=\log\star(x)=\log C + \log\cosh(x/2).
\]

\begin{proposition}
For $m\ge1$,
\[
g^{(m)}(x)=2^{-m}\frac{d^{\,m-1}}{dx^{m-1}}\tanh(x/2).
\]
\end{proposition}

\begin{proof}
Differentiate $\log\cosh(x/2)$ repeatedly and apply the chain rule.  \cite{constantine1996generalized}
\end{proof}


\subsection{Bell polynomial encoding}

Let $B_{n}$ denote the $n$th complete Bell polynomial.

\begin{theorem}
For each $n\ge0$,
\[
\star B_{n}(x)=\frac{d^{n}}{dx^{n}}\star(x)
=\star(x)\,B_{n}\bigl(g'(x),\dots,g^{(n)}(x)\bigr).
\]
\end{theorem}


\subsection{Parity at the origin}

\begin{proposition}\label{prop:parity}
For all $n\ge0$,
\[
\star B_{n}(0)=\begin{cases}
2^{-n}\star(0),&n\ \text{even},\\[4pt]
0,&n\ \text{odd}.
\end{cases}
\]
\end{proposition}

\begin{proof}
$\cosh(x/2)$ is even and its odd derivatives vanish at the origin.
\end{proof}


\section{Formal Integral Transforms and Recurrence Relations}

\subsection{Definitions}

\begin{definition}
For integers $n,m\ge0$ and complex $s$, define
\[
\sigma(s,n,m)=\int_{0}^{\infty} x^{s}\,\star B_{n}(x)\,\star B_{m}(x)\,dx,
\]
\[
\tau(s,n,m)=\bigl[x^{s}\star B_{n}(x)\star B_{m}(x)\bigr]_{0}^{\infty}.
\]
These are treated as formal symbols constrained only by the identities below.
\end{definition}
Then it is clear by our work in Section 2 that:
\begin{theorem}[Relation between $\sigma(s,0,0)$ and the Riemann zeta function]
    For $\Re(s) > 0$, the following identity holds:\cite{andrews1999special}
    \[
    \zeta(s)\Gamma(s+1)(1 - 2^{1-s}) = \frac{1}{4} \sigma(s,0,0),
    \]
    where
    \[
    \sigma(s,0,0) = \int_0^\infty x^s e^x \frac{1}{(e^x + 1)^2} \, dx.
    \]
\end{theorem}

\begin{proof}
    The claim is shown by synthetic division of $(e^x + 1)$ by the Weierstrass product $\star(x)$. 
    Since $\star(x) = C \cosh(x/2)$ for some nonzero constant $C$, we have:
    \[
    \cosh(x/2) = \frac{e^{x/2} + e^{-x/2}}{2}.
    \]
    It follows that:
    \[
    e^x + 1 = 2 e^{x/2} \cosh(x/2).
    \]
    Substituting $\star(x) = C \cosh(x/2)$, we obtain:
    \[
    e^x + 1 = \frac{2}{C} e^{x/2} \star(x).
    \]
    Hence,
    \[
    \frac{1}{(e^x + 1)^2} = \frac{C^2}{4} e^{-x} \frac{1}{\star(x)^2}.
    \]
    Now recall the definition:
    \[
    \sigma(s,0,0) = \int_0^\infty x^s \star B_0 \star B_0 \, dx.
    \]
    Since $B_0 \equiv 1$, this becomes:
    \[
    \sigma(s,0,0) = \int_0^\infty x^s \star(x)^2 \, dx.
    \]
    Substituting the expression for $1/(e^x + 1)^2$ yields:
    \[
    \sigma(s,0,0) = \frac{4}{C^2} \int_0^\infty x^s e^x \frac{1}{(e^x + 1)^2} \, dx.
    \]
    The integral on the right is a known representation related to the Riemann zeta function:
    \[
    \int_0^\infty x^s e^x \frac{1}{(e^x + 1)^2} \, dx = \Gamma(s+1) \zeta(s) (1 - 2^{1-s}), \quad \Re(s) > 0.
    \]
    Therefore,
    \[
    \sigma(s,0,0) = \frac{4}{C^2} \Gamma(s+1) \zeta(s) (1 - 2^{1-s}).
    \]
    Choosing the constant $C = 2$ (which corresponds to a natural normalization of $\star$) gives:
    \[
    \sigma(s,0,0) = 4 \Gamma(s+1) \zeta(s) (1 - 2^{1-s}),
    \]
    or equivalently,
    \[
    \zeta(s)\Gamma(s+1)(1 - 2^{1-s}) = \frac{1}{4} \sigma(s,0,0),
    \]
    as required.
\end{proof}

\subsection{First integration-by-parts identity}

\begin{theorem}\label{thm:ibp1}
For all $s,n,m$,
\[
\sigma(s,n,m)
=\tau(s,n,m)-s\,\sigma(s-1,n,m)-\sigma(s,n+1,m).
\]
\end{theorem}

\begin{proof}
Apply integration by parts formally with
$u=x^{s}\star B_{n}$ and $dv=\star B_{m}\,dx$.
\end{proof}


\subsection{Second integration-by-parts identity}

\begin{theorem}\label{thm:ibp2}
For all $s,n,m$,
\[
\sigma(s,n,m)
=\tau(s,n,m)-s\,\sigma(s-1,n,m)-\sigma(s,n,m+1).
\]
\end{theorem}

\begin{proof}
Apply integration by parts with $u=x^{s}\star B_{m}$ instead.
\end{proof}


\subsection{Consistency and the corrected single-shift identity}

Subtracting Theorems~\ref{thm:ibp2} and \ref{thm:ibp1} yields:

\begin{proposition}[Corrected shift identity]\label{prop:shift}
For all $s,n,m$,
\[
\sigma(s,n+1,m)-\sigma(s,n,m+1)
=s\bigl[\sigma(s-1,n,m+1)-\sigma(s-1,n+1,m)\bigr].
\]
\end{proposition}

The identity will play a key role in the Yang-Baxter analysis.


\subsection{Symmetry}

\begin{proposition}
\[
\sigma(s,n,m)=\sigma(s,m,n),\qquad \tau(s,n,m)=\tau(s,m,n).
\]
\end{proposition}

\begin{proof}
The integrand is symmetric in $n$ and $m$.
\end{proof}


\subsection{Parity and quasi-periodicity}

\begin{proposition}[Parity vanishing]\label{prop:parity_tau}
If $n$ or $m$ is odd, then $\tau(s,n,m)=0$.
\end{proposition}

\begin{proof}
By Proposition~\ref{prop:parity}, $\star B_{n}(0)=0$ for odd $n$ and similarly at $\infty$
formally.
\end{proof}

\begin{proposition}[Two-step quasi-periodicity]\label{prop:qp}
For all $s,n,m$,
\[
\tau(s,n+2,m)=\frac14\,\tau(s,n,m).
\]
\end{proposition}

\begin{proof}
From $\star B_{n+2}(0)=\frac{1}{4}\star B_{n}(0)$ and boundary vanishing for odd indices.
\end{proof}

\begin{corollary}
$\sigma$ satisfies the same quasi-periodicity in its first index:
\[
\sigma(s,n+2,m)=\frac{1}{4}\,\sigma(s,n,m).
\]
\end{corollary}

\begin{proof}
Insert Proposition~\ref{prop:qp} into Theorems~\ref{thm:ibp1}-\ref{thm:ibp2} and argue inductively.
\end{proof}


\section{Construction of a Formal $R$-Matrix}

Let $V=\C^{2}$ with basis $|+\rangle,|-\rangle$.  For $u,v\in\C$, define the
integer index
\[
n(u,v)=\frac{2(u-v)}{i\pi}.
\]

\begin{definition}
Define
\[
A(n)=\tau(s,n,n),\qquad
B(n)=\sigma(s,n,n+1).
\]
The \emph{formal $R$-matrix} is
\[
R(u,v)=
\begin{pmatrix}
A & 0 & 0 & 0\\
0 & B & B & 0\\
0 & B & B & 0\\
0 & 0 & 0 & A
\end{pmatrix},
\]
where $A=A(n(u,v))$ and $B=B(n(u,v))$.
\end{definition}

The parity and quasi-periodicity imply:

\begin{proposition}
$A(n)=0$ for odd $n$, and $A(n+2)=\frac{1}{4}A(n)$; likewise $B(n+2)=\frac{1}{4}B(n)$.
\end{proposition}


\section{Proof of the Yang-Baxter Equation}
To prove the functionals satisfy the Yang-Baxter Equations \cite{yang1967some,baxter1972partition, baxter1982exactly} We must:
\begin{enumerate}
    \item Construct the $R$-Matrix explicitly.
    \item Expressing the products of the \(2\times2\) block matrices explicitly in terms of \(\sigma\) and \(\tau\).
    \item Using the recurrences \ref{thm:ibp1} \ref{thm:ibp2} and the boundary properties in Propositions \ref{prop:parity} and \ref{prop:parity_tau} to reduce both sides of the Yang–Baxter equation to a common form.
    \item Showing that the resulting functional equations are identities modulo the defining relations of \(\sigma\) and \(\tau\).
\end{enumerate}
Let $J$ denote the $2\times2$ matrix
\[
J=\begin{pmatrix}1&1\\1&1\end{pmatrix}.
\]

In the subspace spanned by $|+-\rangle,|-+\rangle$, the $R$-matrix acts as $B(n)J$.
Since $J$ satisfies the Temperley-Lieb relations $J^{2}=2J$, the matrix
Yang-Baxter equation reduces to a scalar condition.

\subsection{Reduction to a scalar triple-product identity}

Write $B(u,v)=B(n(u,v))$. \footnote{The $B$ on the left is definedby the spectral parameters, The $B$ on the right is the previously defined $B(n(u,v))$.} Then the Yang-Baxter equation on the relevant
two-dimensional subspace is equivalent to:
\begin{equation}\label{eq:YBE_scalar}
B(u,v)\,B(u,w)\,B(v,w)=B(v,w)\,B(u,w)\,B(u,v).
\end{equation}
Thus it suffices to prove that the triple product is symmetric in $u,v,w$.

\subsection{Solution of $B(n)$ under the recurrences}

\begin{lemma}\label{lem:Bn_solution}
There exists a function $C(s)$ such that
\[
B(n)=C(s)\cdot 2^{-n}.
\]
\end{lemma}

\begin{proof}
By quasi-periodicity, $B(n+2)=\frac{1}{4}B(n)$, hence $B(n)=K(s)\,2^{-n}$ for some
$K(s)$.  Parity constraints are consistent with this form.
\end{proof}

\begin{proposition}
The triple product in \eqref{eq:YBE_scalar} is symmetric in $u,v,w$.
\end{proposition}

\begin{proof}
Let $n(u,v)=\dfrac{2(u-v)}{i\pi}$.  Then
\[
n(u,v)+n(u,w)+n(v,w)=\frac{2}{i\pi}\bigl[(u-v)+(u-w)+(v-w)\bigr]=0.
\]
Using Lemma~\ref{lem:Bn_solution},
\[
B(u,v)B(u,w)B(v,w)
=C(s)^{3}\,2^{-\,[n(u,v)+n(u,w)+n(v,w)]}=C(s)^{3},
\]
which is symmetric.
\end{proof}

\begin{theorem}[Formal Yang-Baxter equation]
The $R$-matrix defined above satisfies
\[
R_{12}(u,v)R_{13}(u,w)R_{23}(v,w)=
R_{23}(v,w)R_{13}(u,w)R_{12}(u,v)
\]
as a formal identity in $u,v,w$.
\end{theorem}

\begin{proof}
The reduction above shows that all nontrivial components satisfy the scalar
identity \eqref{eq:YBE_scalar}, which holds by the preceding proposition.
\end{proof}


\section{Conclusion}

We have developed a formal calculus built from a regularized Weierstrass
product with the zero set of $\cosh(x/2)$.  Encoding higher derivatives via
Bell polynomials leads to formal transforms $\sigma$ and $\tau$ satisfying a
closed system of integration-by-parts recurrences exhibiting symmetry,
parity, and quasi-periodicity.  These identities are sufficient to construct a
Temperley-Lieb type $R$-matrix that satisfies the Yang-Baxter equation
formally.

The results are purely algebraic.  They isolate structural features that mirror
those of integrable lattice models and suggest that the calculus of classical
Weierstrass products may be linked, at a formal level, to the algebraic
underpinnings of quantum integrability.

Further work may address analytic realizations or operator-theoretic
interpretations of the recurrence system.

\bibliography{referencescopy}

\end{document}

% ****** End of file rmatrixisomorphism.tex ******
